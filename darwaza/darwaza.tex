% Copyright 2016 Abid Hasan Mujtaba
% 
% Licensed under Creative Commons BY-SA 4.0 (https://creativecommons.org/licenses/by-sa/4.0/)


\documentclass{article}

\usepackage{urdu-prose}


\begin{document}

\begin{center} \Large
   دروازہ
\end{center}

سردیوں کے دن تھے اور اسکول سے چھٹیاں بھی۔ کنبہ کے بے چین افراد نے سوچا کہ مری جا کر برف باری کا لطف اٹھایا جائے۔ امی کو ''پریشان`` ہونے کا لقب بلا وجہ نہیں ملا تھا۔ انہوں نے جانے سے صاف انکار کر دیا اور ہر ممکن کوشش کی کہ ہم بچے بھی نہ جا سکیں۔ لیکن ابو جن کا واحد مقصد یہ تھا کہ کوئی انکو لے کر نہ جائے نے ہمیں جانے کی اجازت دے دی۔

تو پھر تین چار گاڑیاں بھر کے بچہ پارٹی ہمراہ ایک عدد خالہ اور خالو کے مری آن پہنچے۔ اندازاً ایک درجن لوگوں کے لئے دو کمرے کرائے پر لئے گئے جو باقی ریسٹ ہائوس سے تھوڑے الگ واقعہ تھے۔ کمروں تک ایک پکڈنڈی جاتے تھی جس کے دونوں طرف پہاڑی کی ڈھلان پر ایک سے دو فُٹ برف موجود تھی۔

سورج ڈھلتے ہی درجَہ حرارت تیزی سے گرنےلگا اور ہم سب لوگ ایک کمرے میں اِکٹھے ہو گئے جس میں لڑکیوں کو سونا تھا۔ رات دیر تک ہلا گلا لگا رہا لیکن کسی سیانے نے ساتھ والے لڑکوں کے کمرے میں ہیٹر چلا چھوڑا تھا تاکہ کمرہ سونے کی خاطر گرم رہے۔

\vspace{\baselineskip}
\begin{flushleft}
   عابد حسن مجتبےٰ\\
   ۵ جولائی ۲۰۱۶\\
   اسلام آباد
\end{flushleft}

\end{document}
