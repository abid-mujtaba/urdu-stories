% Copyright 2016 Abid Hasan Mujtaba
% 
% Licensed under Creative Commons BY-SA 4.0 (https://creativecommons.org/licenses/by-sa/4.0/)


\documentclass{article}

\usepackage{urdu-prose}


\begin{document}

\begin{center} \Large
   دروازہ
\end{center}

سردیوں کے دن تھے اور اسکول سے چھٹیاں بھی۔ کنبہ کے بے چین افراد نے سوچا کہ مری جا کر برف باری کا لطف اٹھایا جائے۔ امی کو ''پریشان`` ہونے کا لقب بلا وجہ نہیں ملا تھا۔ انہوں نے جانے سے صاف انکار کر دیا اور ہر ممکن کوشش کی کہ ہم بچے بھی نہ جا سکیں۔ لیکن ابو جن کا واحد مقصد یہ تھا کہ کوئی انکو لے کر نہ جائے نے ہمیں جانے کی اجازت دے دی۔

تو پھر تین چار گاڑیاں بھر کے بچہ پارٹی ہمراہ ایک عدد خالہ اور خالو کے مری آن پہنچے۔ اندازاً ایک درجن لوگوں کے لئے دو کمرے کرائے پر لئے گئے جو باقی ریسٹ ہائوس سے تھوڑے الگ واقعہ تھے۔ کمروں تک ایک پکڈنڈی جاتے تھی جس کے دونوں طرف پہاڑی کی ڈھلان پر ایک سے دو فُٹ برف موجود تھی۔

سورج ڈھلتے ہی درجَہ حرارت تیزی سے گرنےلگا اور ہم سب لوگ ایک کمرے میں اِکٹھے ہو گئے جس میں لڑکیوں کو سونا تھا۔ رات دیر تک ہلا گلا لگا رہا لیکن کسی سیانے نے ساتھ والے لڑکوں کے کمرے میں ہیٹر چلا چھوڑا تھا تاکہ کمرہ سونے کی خاطر گرم رہے۔

جب جباَی لینے اور اونگنے والوں کی تعداد گانے اور قہقہے لگانے والوں سے قدرے بڑھ گئے تو یہ فیصلہ کیا گیا کہ اب سویا جائے۔ ایک ایک کر کے تمام لڑکے اپنے کمرے میں جانے لگے۔ جب میں اٹھا تو صرف اسّد پیچھے بچا تھا۔ میں نے اس سے کہا کہ جب کمرے میں آنا تو دروازہ کا کنڈا یاد سے بند کر دینا کیونکہ وہ دروازہ خود سے کھل جاتا ہے۔ میں جا کر زمین پر بچھے گدوں پر پڑے کزنز کے بیچ جگہ بنا کر سو گیا۔

رات کو میں مسلسل اٹھتا رہا اور ہر دفعہ یہی محسوس ہوتا کہ ٹھنڈ میں مزید اضافہ ہو گیا ہے۔ گدے اور لحاف کے باوجود دانت بجنے لگے اور ٹھٹرنا ختم نہیں ہو رہا تھا۔ اس قدر سردی میں نے کبھی زندگی میں محسوس نیہں کی تھی۔ دل ہی دل میں سوچا کہ امی ابّو نے نہ آ کر عقل مندی کا فیصلہ کیا۔ کاش میں بھی اسلام آباد میں اپنے گرم بستر میں ہی رہتا۔

کمرے میں سے آنے والی باقی ہلکی ہلکی بے چین آوازوں سے یہی اندازہ ہو رہا تھا کہ سبھی سردی سے سخت پریشان ہیں۔ کانپنے اور ٹھٹرنے کے اس عالم میں اچانک عباس نے مجھ سے کہا ''بھائی۔ دروازہ کھلا ہے۔``

''کیا؟`` میں چونک کر بولا۔ ''کیا مطلب؟ کونسا دروازہ؟``

عباس، جو میرے ساتھ والے لحاف میں دب کر لیٹا تھا خاموش رہا۔ اس کی خاموشی اور ہل چل کی عدم موجودگی سے صاف ظاہر تھا کہ اب وہ اور کوئی بات یا حرکت نہ کرنے کا منجمد ارادہ کر چکہ تھا۔
 
لیکن عباس کے اس ایک جملہ میں کہ ''بھائی۔ دروازدہ کھلا ہے۔`` نے مجھے جھنجوڑ کر رکھ دیا تھا۔ اس کمرے میں صرف ایک ہی دروازہ تھا جس کا ذکر ہو سکتا تھا۔ باہر کا دروازہ جو پکڈنڈی پر کھُتا تھا۔ وہ پک ڈنڈی جس کے دونوں جانب دو دو فُٹ برف پڑی ہوئی تھی۔

میں نے ایک لمبا سانس کھینچا، ہمت باندھی اور ہچکچاتے ہوئے اپنا سَر لحاف سے نکالا۔ کمرے میں ایک ہلکی سی سفید روشنی پھیلے ہوئی تھی حالانکہ کمرے کے اندر کی تمام بتّاں بند تھیں۔ اُس روشنی میں زمین اور بستر پر موجود لوگ لحاف کے بنے ہوئے پہاڑوں کی مانند نظر آ رہے تھے۔ کمرے میں ٹھنڈ کا یہ عالم تھا کہ میرا سانس بھپ بن کر نکل رہا تھا۔

یہ ہرگز لحاف میں سے سر نکالنے کا موقع نہیں تھا لیکن میں عباس کے جملے کے ہاتھوں مجبور تھا۔ ''بھائی۔ دروازدہ کھلا ہے۔`` میں نے مزید ہمت باندھی اور عباس کی طرف سے کروٹ بدل کر دروازہ کی طرف منہ پھیرا۔

دروازہ واقعی کھلا تھا۔ دروازہ کے باہر ایک سفید ٹیوب لائٹ تھی۔ اُس کی روشنی باور پڑی ہوئی برف سے ٹکڑا کر کمرے میں داخل ہو رہی تھی۔ ایک ہی منظر میں مجھے نیم اندھیرے میں ابھرے ہوئے ٹھٹرتے لحاف اور چمکتی ہوئی برف دکھائی دے رہی تھی۔

چند لمحوں کے لئے میرا دماغ اِس منظر سے دنگ ہو کر رہ گیا۔ عقل تھی محوِ تماشہ لبِ برف ابھی۔ جب ہوش میں آیا تو زہن میں دو خیال آئے۔ پہلا تو یہ کہ میں نے اُس الو کے پٹھے اسّد کو دروازہ کی کنڈی لگانے کو خاص طور پر کہا تھا۔ اُس کی لاپرواہی کی وجہ سے ہم سب جانے کب سے ٹھٹر رہے تھے۔

اور دوسرا خیال یہ کہ جب عباس نے یہ دیکھ ہی لیا تھا کہ دروازہ کھلا ہے تو مجھے جگانے کی کیا ضرورت تھی۔ خود جا کر دروازہ بند کر دیتا۔ لیکن چھوٹے بھائیوں کی خباصت سے کون ناواقف ہے۔ صاف ظاہر تھا کہ وہ کمینہ چاہتا تھا کہ میں اِس بے انتہا ٹھنڈ میں اُٹھ کر دروازہ بند کروں اور وہ لحاف میں چھپ کر لیٹا رہے۔



\vspace{\baselineskip}
\begin{flushleft}
   عابد حسن مجتبےٰ\\
   ۵ جولائی ۲۰۱۶\\
   اسلام آباد
\end{flushleft}

\end{document}
